\documentclass[11pt,letterpaper]{article}
\usepackage{ctable}

\renewcommand{\thetable}{S\arabic{table}}
\renewcommand{\thesection}{S\arabic{section}}

\setlength{\parindent}{0em}
\setlength{\parskip}{1em}

\begin{document}
\title{Scaffold-Based Analytics: Enabling Hit-to-Lead Decisions by
  Visualizing Chemical Series Linked Across Large Datasets}
\author{Deepak Bandyopadhyay$^{\dagger}$, Constantine Kreatsoulas$^{\dagger}$
  Pat G. Brady$^{\dagger}$, \\
  Joseph Boyer$^{\dagger}$, Zangdong He$^{\dagger}$, 
  Genaro Scavello Jr.$^{\dagger}$,\\
  Dac-Trung Nguyen$^{\ddagger}$,
  Tyler Peryea$^{\ddagger}$,
  Rajarshi Guha$^{\ddagger}$,
  Ajit Jadhav$^{\ddagger}$ \\
\\
$^{\dagger}$ GlaxoSmithKline, 1250 S. Collegeville Rd, Collegeville,
PA 19426 \\
\\
$^{\ddagger}$ National Center for Advancing Translational Science, \\ 9800 Medical Center Drive, Rockville, MD 20850}
\maketitle

\section{Scaffold metrics}
\label{sec:scaffold-metrics}

The NCATS R-group tool is designed to fragment a collection of
molecules. In addition to the fragmentation procedure it computes a
series of scaffold metrics, described in Table
\ref{table:scaffoldfilecolumns}. In this section we provide some
details about the \textit{ScaffoldScore} and \textit{Complexity}
metrics.

The \textit{ScaffoldScore} is an empirical metric designed to
summarize a scaffold (or more generally, a fragment) and the compounds
containing the scaffold. Specifically, we define it as
\begin{equation}
  \label{eq:1}
  S = foo
\end{equation}

The \textit{Complexity} metric is an implemention of the empirical
complexity metric described by Barone and Channon \cite{Barone2001}.
\textit{Complexity} can be used to prune away scaffolds that are too
simple, by setting a cutoff such as 100.



\begin{table}[h]
  \centering
  \begin{tabular}[h]{lp{0.75\linewidth}}
    \hline
    \textbf{Column Name} & \textbf{Description} \\
    \hline
    ScaffoldID & Numeric scaffold identifier. Each scaffold occurs only
    once, and data columns are aggregated for all molecules containing the
    scaffold \\
    Structure & Scaffold SMILES without R-groups attached \\
    RgroupLabels & A comma separated list of R-group labels for
    all R-groups associated with the scaffold \\
    ScaffoldScore & XXX \\
    Complexity & A number that captures increasing size and complexity of
    scaffolds. See Section \ref{sec:scaffold-metrics}.  \\
    Count & Number of molecules that share this scaffold \\
    \hline
  \end{tabular}
  \caption{A description of the fixed columns of the scaffold file generated
    by the NCATS R-group tool. Additional columns may be present which
    correspond to aggregated property columns.}
  \label{table:scaffoldfilecolumns}
\end{table}

\begin{table}[h]
  \centering
  \begin{tabular}[h]{lp{0.75\linewidth}}
    \hline
    \textbf{Column Name} & \textbf{Description} \\
    \hline
    ScaffoldID & Numeric scaffold identifier (corresponding to
  the {\it ScaffoldID} column in the scaffold file, Table \ref{table:scaffoldfilecolumns}) \\
  MolID & Numeric or text molecule identifier (name). Each molecule is
  repeated once for each scaffold that it occurs in \\
  Structure & Molecule structure in SMILES format \\
  $R_1, \ldots, R_n$ & R-group SMILES, with \*-atoms at attachment
  points. By default we limit to $n = 21$ \\
  \hline
  \end{tabular}
  \caption{A description of the columns in the R-group decomposition
    file generated by the NCATS R-group tool.}
\end{table}

\newpage

\bibliographystyle{unsrt}
\bibliography{bibliography}

\end{document}
